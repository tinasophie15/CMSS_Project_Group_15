\documentclass[12pt,a4paper,twocolumn]{article}
\usepackage[margin=2cm,columnsep=0.8cm]{geometry} 
\usepackage{setspace}                    
\usepackage{authblk}                      
\usepackage{graphicx}                    
\usepackage{subcaption}                 
\usepackage{float}                       
\usepackage{booktabs}                   
\usepackage{amsmath,amsfonts,amssymb}   
\usepackage{siunitx}                  
\usepackage[round,sort&compress]{natbib}
\usepackage{hyperref}                    
\usepackage{caption}                    
\usepackage{fancyhdr}                   
\usepackage{xcolor}                    
\usepackage{libertine}
\captionsetup{font=small,labelfont=bf}
\pagestyle{fancy}

\fancyhf{}
\fancyhead[L]{\textit{Running Title}}  % Edit the running title (short title)
\fancyhead[R]{\thepage}

% For incoming students, feel free to use your home university here
\title{Full Title of the Article} % Pick your title (you may go creative here)
\author[1]{First Author} % Each author should be listed
\author[2]{Second Author}
\affil[1]{Graz University of Technology, Graz, Austria}
\affil[2]{University of Graz, Graz, Austria}
\date{\today}
\begin{document}
\maketitle

\begin{abstract}
    Your abstract goes here. It should be a single paragraph of 150--250 words summarizing the report’s context, aims, methods, results, and conclusions.
    The entire paper should not exceed 6 pages (excluding limitations, ethical considerations, references, appendix).
    Please do not forget to fill out the contributions for each team member in the table.
\end{abstract}

\section{Introduction}
\label{sec:intro}
State the context and objectives of the work, optionally provide a short background.
In scientific writing it is common to use "we", even for single author works.
Often contractions are avoided in scientific writing.

\section{Related Work}
Please cite relevant literature.
One important paper for illustration proposes the attention mechanism of the Transformer architecture~\citep{vaswani2017attention}.
Or use \citet{vaswani2017attention} to use the authors as part of the sentence.

\section{Materials and Methods}
\label{sec:methods}
Describe the materials (datasets), and analytical methods used.
Consider using pseudocode or mathematical notation to aid communication of your work.
The reader should be able to reproduce the results.

\section{Results}
\label{sec:results}
Present the evaluation results you gained (objective results). 
Consider using tables and charts for the report.
Of course, baselines are helpful for the reader to assess the performance of the method.

\section{Discussion}
\label{sec:discussion}
Interpret and discuss your results (could be a more subjective results).
Describe your findings here, what can we learn from the work.

\section{Conclusion}
\label{sec:conclusion}
Summarize the key findings and implications (design your report that one get the main insights from reading abstract/introduction/conclusions and glancing at the illustrations). 
Suggest future research (very briefly).


% Everything up here counts for the page limit 
\newpage

\section*{Limitations}
List potential limitations of your work, e.g., only English language.

\section*{Ethical Considerations}
Please consider how your work could potentially be used and may cause some harm.
Also, sustainability aspects can be reported here, e.g., how much CO2e does you approach require.

\section*{Acknowledgments}
Briefly acknowledge people, funding sources, or institutions.

\section*{Contributions}
% Add here a list of the individual contributions per team member
\begin{table}[h!]
    \centering
    \caption{List of contributions per team member.}
    \label{tab:contributions}
    \begin{tabular}{lp{4cm}}
        \toprule
        \textbf{Team Member} & \textbf{Contribution} \\
        \midrule
        Name            & Did all of the work.             \\
        Another Name    & Did not even care to show up.    \\
        \bottomrule
    \end{tabular}
\end{table}

\bibliographystyle{plainnat}
\bibliography{bibliography}

\appendix

\section{Appendix - Overview}

The appendix can be used to add details, especially implementation aspects, or added evaluations.
There is no page limit on the appendix.
You may also report approaches that you tried, but did not work out.
Additional examples can be reported, or prompts being used for generative AI.

\section{Appendix - Usage of AI}
More details on the usage of AI: \href{https://www.tugraz.at/fileadmin/Studierende\_und\_Bedienstete/Information/Unsere\_TU\_Graz/Lehre\_an\_der\_TU\_Graz/Zitiervorschlaege_KI.pdf}{Zitiervorschlaege AI}

\section{Appendix - Figure and Table Examples}
Examples how to use figures, see Figure~\ref{fig:example} and tables, see Table~\ref{tab:contributions}.
In double, just let Latex layout the illustrations for you, or position them at the top or bottom of a page.
Is is common to capitalise nouns that are followed by numbers, as they are considered names (proper noun), e.g., Page 4.

\begin{figure}
    \centering
        \includegraphics[width=0.7\linewidth]{example-image}
    \caption{Example figure caption. Please consider to explain to the reader, what is depicted}
    \label{fig:example}
\end{figure}


\end{document}
